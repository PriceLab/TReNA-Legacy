\nonstopmode{}
\documentclass[a4paper]{book}
\usepackage[times,inconsolata,hyper]{Rd}
\usepackage{makeidx}
\usepackage[utf8,latin1]{inputenc}
% \usepackage{graphicx} % @USE GRAPHICX@
\makeindex{}
\begin{document}
\chapter*{}
\begin{center}
{\textbf{\huge Package `TReNA'}}
\par\bigskip{\large \today}
\end{center}
\begin{description}
\raggedright{}
\item[Type]\AsIs{Package}
\item[Title]\AsIs{Fit transcriptional regulatory networks using gene expression,
priors, machine learning}
\item[Version]\AsIs{0.99.66}
\item[Date]\AsIs{2017-02-09}
\item[Author]\AsIs{Seth Ament }\email{seth.ament@systemsbiology.org}\AsIs{, Paul Shannon }\email{pshannon@systemsbioloyg.org}\AsIs{, Matthew Richards }\email{mrichard@systemsbiology.org}\AsIs{}
\item[Maintainer]\AsIs{Seth Ament }\email{seth.ament@systemsbiology.org}\AsIs{, Paul Shannon }\email{pshannon@systemsbioloyg.org}\AsIs{, Matthew Richards }\email{mrichard@systemsbiology.org}\AsIs{}
\item[Depends]\AsIs{R (>= 3.2.3), glmnet (>= 2.0.3), randomForest, vbsr,
RPostgreSQL , GenomicRanges , foreach , doParallel}
\item[Suggests]\AsIs{RUnit}
\item[Description]\AsIs{Methods for reconstructing transcriptional regulatory networks, especially in species
for which genome-wide TF binding site information is available.}
\item[License]\AsIs{GPL-3}
\item[Collate]\AsIs{'Solver.R' 'BayesSpikeSolver.R' 'Filter.R' 'FootprintFinder.R'
'LassoSolver.R' 'NullFilter.R' 'PearsonSolver.R'
'RandomForestSolver.R' 'SpearmanSolver.R' 'TReNA.R'
'VarianceFilter.R' 'runTReNA.R'}
\item[RoxygenNote]\AsIs{6.0.1}
\end{description}
\Rdcontents{\R{} topics documented:}
\inputencoding{utf8}
\HeaderA{BayesSpikeSolver}{Designate Bayes Spike as the TReNA Solver and Solve}{BayesSpikeSolver}
%
\begin{Description}\relax
Designate Bayes Spike as the TReNA Solver and Solve
\end{Description}
%
\begin{Usage}
\begin{verbatim}
BayesSpikeSolver(mtx.assay = matrix(), quiet = TRUE)
\end{verbatim}
\end{Usage}
%
\begin{Arguments}
\begin{ldescription}
\item[\code{mtx.assay}] An assay matrix of gene expression data
\end{ldescription}
\end{Arguments}
%
\begin{Value}
A Solver class object with Bayes Spike as the solver
\end{Value}
\inputencoding{utf8}
\HeaderA{BayesSpikeSolver-class}{An S4 class to represent a Bayes Spike solver}{BayesSpikeSolver.Rdash.class}
\aliasA{.BayesSpikeSolver}{BayesSpikeSolver-class}{.BayesSpikeSolver}
%
\begin{Description}\relax
An S4 class to represent a Bayes Spike solver
\end{Description}
\inputencoding{utf8}
\HeaderA{Filter-class}{An S4 class to represent a filter}{Filter.Rdash.class}
\aliasA{.Filter}{Filter-class}{.Filter}
%
\begin{Description}\relax
An S4 class to represent a filter
\end{Description}
%
\begin{Arguments}
\begin{ldescription}
\item[\code{mtx.assay}] An assay matrix of gene expression data
\end{ldescription}
\end{Arguments}
\inputencoding{utf8}
\HeaderA{FootprintFinder-class}{Filter genes based on footprints}{FootprintFinder.Rdash.class}
\aliasA{.FootprintFinder}{FootprintFinder-class}{.FootprintFinder}
%
\begin{Description}\relax
Filter genes based on footprints
\end{Description}
%
\begin{Arguments}
\begin{ldescription}
\item[\code{genome.database.uri}] The address of a genome database for use in filtering

\item[\code{project.database.uri}] The address of a project database for use in filtering
\end{ldescription}
\end{Arguments}
%
\begin{Value}
An object of the Filter class that reduces a list of genes to a subset prior to forming a TReNA object
\end{Value}
\inputencoding{utf8}
\HeaderA{getAssayData,Solver-method}{Get Assay Data from Solver}{getAssayData,Solver.Rdash.method}
%
\begin{Description}\relax
Get Assay Data from Solver
\end{Description}
%
\begin{Usage}
\begin{verbatim}
## S4 method for signature 'Solver'
getAssayData(obj)
\end{verbatim}
\end{Usage}
%
\begin{Value}
The assay matrix of gene expression data associated with a Solver object
\end{Value}
\inputencoding{utf8}
\HeaderA{getCandidates}{Get candidate genes using the selected filter}{getCandidates}
\aliasA{Filter}{getCandidates}{Filter}
%
\begin{Description}\relax
Get candidate genes using the selected filter
\end{Description}
%
\begin{Usage}
\begin{verbatim}
Filter(mtx.assay = matrix(), quiet = TRUE)
\end{verbatim}
\end{Usage}
%
\begin{Value}
A vector containing all genes selected as candidates by the filter
\end{Value}
\inputencoding{utf8}
\HeaderA{getCandidates,NullFilter-method}{Get candidate genes using the null filter}{getCandidates,NullFilter.Rdash.method}
\aliasA{getCandidates-NullFilter}{getCandidates,NullFilter-method}{getCandidates.Rdash.NullFilter}
%
\begin{Description}\relax
Get candidate genes using the null filter
\end{Description}
%
\begin{Usage}
\begin{verbatim}
## S4 method for signature 'NullFilter'
getCandidates(obj)
\end{verbatim}
\end{Usage}
%
\begin{Value}
A vector containing all genes in the assay matrix
\end{Value}
\inputencoding{utf8}
\HeaderA{getCandidates,VarianceFilter-method}{Get candidate genes using the variance filter}{getCandidates,VarianceFilter.Rdash.method}
\aliasA{getCandidates-VarianceFilter}{getCandidates,VarianceFilter-method}{getCandidates.Rdash.VarianceFilter}
%
\begin{Description}\relax
Get candidate genes using the variance filter
\end{Description}
%
\begin{Usage}
\begin{verbatim}
## S4 method for signature 'VarianceFilter'
getCandidates(obj, target.gene)
\end{verbatim}
\end{Usage}
%
\begin{Arguments}
\begin{ldescription}
\item[\code{target.gene}] A designated target gene that should be part of the mtx.assay data
\end{ldescription}
\end{Arguments}
%
\begin{Value}
A vector containing all genes with variances less than the target gene
\end{Value}
\inputencoding{utf8}
\HeaderA{getFootprintsForTF}{Get Footprints for a TF}{getFootprintsForTF}
%
\begin{Description}\relax
Get all footprints from the project database for a specified transcription factor
\end{Description}
%
\begin{Usage}
\begin{verbatim}
getFootprintsForTF(obj, tf)
\end{verbatim}
\end{Usage}
%
\begin{Arguments}
\begin{ldescription}
\item[\code{tf}] A transcription factor
\end{ldescription}
\end{Arguments}
\inputencoding{utf8}
\HeaderA{getGenePromoterRegions}{Get Gene Promoter Regions}{getGenePromoterRegions}
%
\begin{Description}\relax
Get the promoter regions for a list of genes. Region sizes can be tuned by specifying
the positions upstream and downstream of the gene that bound the region.
\end{Description}
%
\begin{Usage}
\begin{verbatim}
getGenePromoterRegions(obj, genelist, size.upstream = 10000,
  size.downstream = 10000)
\end{verbatim}
\end{Usage}
%
\begin{Arguments}
\begin{ldescription}
\item[\code{genelist}] A gene list

\item[\code{size.upstream}] Number of basepairs to grab upstream of each gene (default = 10000)

\item[\code{size.downstream}] Number of basepairs to grab downstream of each gene (default = 10000)
\end{ldescription}
\end{Arguments}
\inputencoding{utf8}
\HeaderA{getSolverName,BayesSpikeSolver-method}{Get Bayes Spike Solver name}{getSolverName,BayesSpikeSolver.Rdash.method}
%
\begin{Description}\relax
Get Bayes Spike Solver name
\end{Description}
%
\begin{Usage}
\begin{verbatim}
## S4 method for signature 'BayesSpikeSolver'
getSolverName(obj)
\end{verbatim}
\end{Usage}
%
\begin{Value}
"BayesSpikeSolver"
\end{Value}
\inputencoding{utf8}
\HeaderA{getSolverName,LassoSolver-method}{Get Lasso Solver name}{getSolverName,LassoSolver.Rdash.method}
%
\begin{Description}\relax
Get Lasso Solver name
\end{Description}
%
\begin{Usage}
\begin{verbatim}
## S4 method for signature 'LassoSolver'
getSolverName(obj)
\end{verbatim}
\end{Usage}
%
\begin{Value}
"LassoSolver"
\end{Value}
%
\begin{Examples}
\begin{ExampleCode}
solver <- LassoSolver()
getSolverName(solver)
\end{ExampleCode}
\end{Examples}
\inputencoding{utf8}
\HeaderA{getSolverName,PearsonSolver-method}{Get Pearson Solver name}{getSolverName,PearsonSolver.Rdash.method}
%
\begin{Description}\relax
Get Pearson Solver name
\end{Description}
%
\begin{Usage}
\begin{verbatim}
## S4 method for signature 'PearsonSolver'
getSolverName(obj)
\end{verbatim}
\end{Usage}
%
\begin{Value}
"PearsonSolver
\end{Value}
\inputencoding{utf8}
\HeaderA{getSolverName,RandomForestSolver-method}{Get Random Forest Solver name}{getSolverName,RandomForestSolver.Rdash.method}
%
\begin{Description}\relax
Get Random Forest Solver name
\end{Description}
%
\begin{Usage}
\begin{verbatim}
## S4 method for signature 'RandomForestSolver'
getSolverName(obj)
\end{verbatim}
\end{Usage}
%
\begin{Value}
"RandomForestSolver"
\end{Value}
\inputencoding{utf8}
\HeaderA{getSolverName,SpearmanSolver-method}{Get Spearman Solver Name}{getSolverName,SpearmanSolver.Rdash.method}
%
\begin{Description}\relax
Get Spearman Solver Name
\end{Description}
%
\begin{Usage}
\begin{verbatim}
## S4 method for signature 'SpearmanSolver'
getSolverName(obj)
\end{verbatim}
\end{Usage}
%
\begin{Value}
"SpearmanSolver"
\end{Value}
\inputencoding{utf8}
\HeaderA{LassoSolver}{Create a Solver class object using the LASSO solver}{LassoSolver}
%
\begin{Description}\relax
Create a Solver class object using the LASSO solver
\end{Description}
%
\begin{Usage}
\begin{verbatim}
LassoSolver(mtx.assay = matrix(), quiet = TRUE)
\end{verbatim}
\end{Usage}
%
\begin{Arguments}
\begin{ldescription}
\item[\code{mtx.assay}] An assay matrix of gene expression data
\end{ldescription}
\end{Arguments}
%
\begin{Value}
A Solver class object with LASSO as the solver
\end{Value}
%
\begin{Examples}
\begin{ExampleCode}
solver <- LassoSolver()
\end{ExampleCode}
\end{Examples}
\inputencoding{utf8}
\HeaderA{LassoSolver-class}{An S4 class to represent a LASSO solver}{LassoSolver.Rdash.class}
\aliasA{.LassoSolver}{LassoSolver-class}{.LassoSolver}
%
\begin{Description}\relax
An S4 class to represent a LASSO solver
\end{Description}
\inputencoding{utf8}
\HeaderA{NullFilter}{Define an object of class Null Filter}{NullFilter}
%
\begin{Description}\relax
Define an object of class Null Filter
\end{Description}
%
\begin{Usage}
\begin{verbatim}
NullFilter(mtx.assay = matrix(), quiet = TRUE)
\end{verbatim}
\end{Usage}
%
\begin{Arguments}
\begin{ldescription}
\item[\code{mtx.assay}] An assay matrix of gene expression data
\end{ldescription}
\end{Arguments}
%
\begin{Value}
An object of the Null Filter class
\end{Value}
\inputencoding{utf8}
\HeaderA{NullFilter-class}{Apply a null filter}{NullFilter.Rdash.class}
\aliasA{.NullFilter}{NullFilter-class}{.NullFilter}
%
\begin{Description}\relax
Apply a null filter
\end{Description}
%
\begin{Arguments}
\begin{ldescription}
\item[\code{mtx.assay}] An assay matrix of gene expression data
\end{ldescription}
\end{Arguments}
%
\begin{Value}
An object of class NullFilter
\end{Value}
\inputencoding{utf8}
\HeaderA{PearsonSolver}{Create a Solver class object using  Pearson correlation coefficients as the solver}{PearsonSolver}
%
\begin{Description}\relax
Create a Solver class object using  Pearson correlation coefficients as the solver
\end{Description}
%
\begin{Usage}
\begin{verbatim}
PearsonSolver(mtx.assay = matrix(), quiet = TRUE)
\end{verbatim}
\end{Usage}
%
\begin{Arguments}
\begin{ldescription}
\item[\code{mtx.assay}] An assay matrix of gene expression data
\end{ldescription}
\end{Arguments}
\inputencoding{utf8}
\HeaderA{RandomForestSolver}{Create a Solver class object using the Random Forest Solver}{RandomForestSolver}
%
\begin{Description}\relax
Create a Solver class object using the Random Forest Solver
\end{Description}
%
\begin{Usage}
\begin{verbatim}
RandomForestSolver(mtx.assay = matrix(), quiet = TRUE)
\end{verbatim}
\end{Usage}
%
\begin{Arguments}
\begin{ldescription}
\item[\code{mtx.assay}] An assay matrix of gene expression data
\end{ldescription}
\end{Arguments}
%
\begin{Value}
A Solver class object with Random Forest as the solver
\end{Value}
\inputencoding{utf8}
\HeaderA{RandomForestSolver-class}{An S4 class to represent a Random Forest solver}{RandomForestSolver.Rdash.class}
\aliasA{.RandomForestSolver}{RandomForestSolver-class}{.RandomForestSolver}
%
\begin{Description}\relax
An S4 class to represent a Random Forest solver
\end{Description}
\inputencoding{utf8}
\HeaderA{rescalePredictorWeights}{Rescale Bayes Spike Predictor Weights}{rescalePredictorWeights}
\methaliasA{rescalePredictorWeights.BayesSpikeSolver}{rescalePredictorWeights}{rescalePredictorWeights.BayesSpikeSolver}
%
\begin{Description}\relax
Rescale Bayes Spike Predictor Weights
\end{Description}
%
\begin{Usage}
\begin{verbatim}
rescalePredictorWeights(obj, rawValue.min, rawValue.max, rawValues)
\end{verbatim}
\end{Usage}
%
\begin{Arguments}
\begin{ldescription}
\item[\code{rawValue.min}] The minimum value of the raw expression values

\item[\code{rawValue.max}] The maximum value of the raw expression values

\item[\code{rawValues}] A matrix of raw expression values
\end{ldescription}
\end{Arguments}
%
\begin{Value}
A matrix of the raw values re-scaled using the minimum and maximum values
\end{Value}
\inputencoding{utf8}
\HeaderA{rescalePredictorWeights,LassoSolver-method}{Rescale LASSO Predictor Weights}{rescalePredictorWeights,LassoSolver.Rdash.method}
\aliasA{rescalePredictorWeights.LassoSolver}{rescalePredictorWeights,LassoSolver-method}{rescalePredictorWeights.LassoSolver}
%
\begin{Description}\relax
Rescale LASSO Predictor Weights
\end{Description}
%
\begin{Usage}
\begin{verbatim}
## S4 method for signature 'LassoSolver'
rescalePredictorWeights(obj, rawValue.min, rawValue.max,
  rawValues)
\end{verbatim}
\end{Usage}
%
\begin{Arguments}
\begin{ldescription}
\item[\code{rawValue.min}] The minimum value of the raw expression values

\item[\code{rawValue.max}] The maximum value of the raw expression values

\item[\code{rawValues}] A matrix of raw expression values
\end{ldescription}
\end{Arguments}
%
\begin{Value}
A matrix of the raw values re-scaled using the minimum and maximum values
\end{Value}
\inputencoding{utf8}
\HeaderA{run,BayesSpikeSolver-method}{Run the Bayes Spike Solver}{run,BayesSpikeSolver.Rdash.method}
\aliasA{run.BayesSpikeSolver}{run,BayesSpikeSolver-method}{run.BayesSpikeSolver}
%
\begin{Description}\relax
Given a TReNA object with Bayes Spike as the solver, use the \code{\LinkA{vbsr}{vbsr}} function to estimate coefficients
for each transcription factor as a predictor of the target gene's expression level.
\end{Description}
%
\begin{Usage}
\begin{verbatim}
## S4 method for signature 'BayesSpikeSolver'
run(obj, target.gene, tfs, tf.weights = rep(1,
  length(tfs), extraArgs = list()))
\end{verbatim}
\end{Usage}
%
\begin{Arguments}
\begin{ldescription}
\item[\code{target.gene}] A designated target gene that should be part of the mtx.assay data

\item[\code{tfs}] The designated set of transcription factors that could be associated with the target gene.

\item[\code{tf.weights}] A set of weights on the transcription factors (default = rep(1, length(tfs)))

\item[\code{extraArgs}] Modifiers to the Bayes Spike solver
\end{ldescription}
\end{Arguments}
%
\begin{Value}
A data frame containing the coefficients relating the target gene to each transcription factor, plus other fit parameters.
\end{Value}
%
\begin{SeeAlso}\relax
\code{\LinkA{vbsr}{vbsr}}
\end{SeeAlso}
\inputencoding{utf8}
\HeaderA{run,LassoSolver-method}{Run the LASSO Solver}{run,LassoSolver.Rdash.method}
\aliasA{run.LassoSolver}{run,LassoSolver-method}{run.LassoSolver}
%
\begin{Description}\relax
Given a TReNA object with LASSO as the solver, use the \code{\LinkA{glmnet}{glmnet}} function to estimate coefficients
for each transcription factor as a predictor of the target gene's expression level.
\end{Description}
%
\begin{Usage}
\begin{verbatim}
## S4 method for signature 'LassoSolver'
run(obj, target.gene, tfs, tf.weights = rep(1,
  length(tfs)), extraArgs = list())
\end{verbatim}
\end{Usage}
%
\begin{Arguments}
\begin{ldescription}
\item[\code{target.gene}] A designated target gene that should be part of the mtx.assay data

\item[\code{tfs}] The designated set of transcription factors that could be associated with the target gene.

\item[\code{tf.weights}] A set of weights on the transcription factors (default = rep(1, length(tfs)))

\item[\code{extraArgs}] Modifiers to the LASSO solver
\end{ldescription}
\end{Arguments}
%
\begin{Value}
A data frame containing the coefficients relating the target gene to each transcription factor, plus other fit parameters.
\end{Value}
%
\begin{SeeAlso}\relax
\code{\LinkA{glmnet}{glmnet}}
\end{SeeAlso}
\inputencoding{utf8}
\HeaderA{run,PearsonSolver-method}{Run the Pearson Solver}{run,PearsonSolver.Rdash.method}
\aliasA{run.PearsonSolver}{run,PearsonSolver-method}{run.PearsonSolver}
%
\begin{Description}\relax
Given a TReNA object with Pearson as the solver, use the \code{\LinkA{cor}{cor}} function to
estimate coefficients for each transcription factor as a perdictor of the target gene's expression level
\end{Description}
%
\begin{Usage}
\begin{verbatim}
## S4 method for signature 'PearsonSolver'
run(obj, target.gene, tfs, tf.weights = rep(1,
  length(tfs), extraArgs = list()))
\end{verbatim}
\end{Usage}
%
\begin{Arguments}
\begin{ldescription}
\item[\code{target.gene}] A designated target gene that should be part of the mtx.assay data

\item[\code{tfs}] The designated set of transcription factors that could be associated with the target gene.
\end{ldescription}
\end{Arguments}
%
\begin{Value}
fit The set of Pearson Correlation Coefficients between each transcription factor and the target gene.
\end{Value}
%
\begin{SeeAlso}\relax
\code{\LinkA{cor}{cor}}
\end{SeeAlso}
\inputencoding{utf8}
\HeaderA{run,RandomForestSolver-method}{Run the Random Forest Solver}{run,RandomForestSolver.Rdash.method}
\aliasA{run.RandomForestSolver}{run,RandomForestSolver-method}{run.RandomForestSolver}
%
\begin{Description}\relax
Run the Random Forest Solver
\end{Description}
%
\begin{Usage}
\begin{verbatim}
## S4 method for signature 'RandomForestSolver'
run(obj, target.gene, tfs, tf.weights = rep(1,
  length(tfs), extraArgs = list()))
\end{verbatim}
\end{Usage}
%
\begin{Arguments}
\begin{ldescription}
\item[\code{target.gene}] A designated target gene that should be part of the mtx.assay data

\item[\code{tfs}] The designated set of transcription factors that could be associated with the target gene.

\item[\code{tf.weights}] A set of weights on the transcription factors (default = rep(1, length(tfs)))

\item[\code{extraArgs}] Modifiers to the Random Forest solver
\end{ldescription}
\end{Arguments}
%
\begin{Value}
A list containing various parameters of the Random Forest fit.
\end{Value}
\inputencoding{utf8}
\HeaderA{run,SpearmanSolver-method}{Run the Spearman Solver}{run,SpearmanSolver.Rdash.method}
\aliasA{run.SpearmanSolver}{run,SpearmanSolver-method}{run.SpearmanSolver}
%
\begin{Description}\relax
Given a TReNA object with Spearman as the solver, use the \code{\LinkA{cor}{cor}} function with
\code{method = "spearman"} to esimate coefficients for each transcription factor as a predictor of the target
gene's expression level
\end{Description}
%
\begin{Usage}
\begin{verbatim}
## S4 method for signature 'SpearmanSolver'
run(obj, target.gene, tfs, tf.weights = rep(1,
  length(tfs), extraArgs = list()))
\end{verbatim}
\end{Usage}
%
\begin{Arguments}
\begin{ldescription}
\item[\code{target.gene}] A designated target gene that should be part of the mtx.assay data

\item[\code{tfs}] The designated set of transcription factors that could be associated with the target gene.
\end{ldescription}
\end{Arguments}
%
\begin{Value}
fit The set of Spearman Correlation Coefficients between each transcription factor and the target gene.
\end{Value}
%
\begin{SeeAlso}\relax
\code{\LinkA{cor}{cor}}
\end{SeeAlso}
\inputencoding{utf8}
\HeaderA{solve,TReNA-method}{Solve the TReNA object}{solve,TReNA.Rdash.method}
%
\begin{Description}\relax
Solve the TReNA object
\end{Description}
%
\begin{Usage}
\begin{verbatim}
## S4 method for signature 'TReNA'
solve(obj, target.gene, tfs, tf.weights = rep(1,
  length(tfs)), extraArgs = list())
\end{verbatim}
\end{Usage}
%
\begin{Arguments}
\begin{ldescription}
\item[\code{target.gene}] A designated target gene that should be part of the mtx.assay data

\item[\code{tfs}] The designated set of transcription factors that could be associated with the target gene.

\item[\code{tf.weights}] A set of weights on the transcription factors (default = rep(1, length(tfs)))

\item[\code{extraArgs}] Modifiers to the Bayes Spike solver
\end{ldescription}
\end{Arguments}
%
\begin{Value}
A data frame containing coefficients relating the target gene to each transcription factor
\end{Value}
\inputencoding{utf8}
\HeaderA{Solver}{Define an object of class Solver}{Solver}
%
\begin{Description}\relax
Define an object of class Solver
\end{Description}
%
\begin{Usage}
\begin{verbatim}
Solver(mtx.assay = matrix(), quiet = TRUE)
\end{verbatim}
\end{Usage}
%
\begin{Arguments}
\begin{ldescription}
\item[\code{mtx.assay}] An assay matrix of gene expression data
\end{ldescription}
\end{Arguments}
%
\begin{Value}
An object of the Solver class
\end{Value}
\inputencoding{utf8}
\HeaderA{Solver-class}{An S4 class to represent a solver}{Solver.Rdash.class}
\aliasA{.Solver}{Solver-class}{.Solver}
%
\begin{Description}\relax
An S4 class to represent a solver
\end{Description}
%
\begin{Section}{Slots}

\begin{description}

\item[\code{mtx.assay}] An assay matrix of gene expression data

\item[\code{quiet}] A logical element indicating whether the solver should produce output

\item[\code{state}] Environment

\end{description}
\end{Section}
\inputencoding{utf8}
\HeaderA{SpearmanSolver}{Create a Solver class object using Spearman correlation coefficients as the solver}{SpearmanSolver}
%
\begin{Description}\relax
Create a Solver class object using Spearman correlation coefficients as the solver
\end{Description}
%
\begin{Usage}
\begin{verbatim}
SpearmanSolver(mtx.assay = matrix(), quiet = TRUE)
\end{verbatim}
\end{Usage}
%
\begin{Arguments}
\begin{ldescription}
\item[\code{mtx.assay}] An assay matrix of gene expression data
\end{ldescription}
\end{Arguments}
%
\begin{Value}
A Solver class object with Spearman correlation coefficients as the solver
\end{Value}
\inputencoding{utf8}
\HeaderA{TReNA}{TReNA:  Fit transcriptional regulatory networks using transcription factor binding sites and lasso regression}{TReNA}
\aliasA{TReNA-package}{TReNA}{TReNA.Rdash.package}
\keyword{networks}{TReNA}
%
\begin{Description}\relax
Methods for reconstructing transcriptional regulatory networks, especially in species
for which genome-wide TF binding site information is available.

\end{Description}
\inputencoding{utf8}
\HeaderA{TReNA-class}{An S4 class to represent a TReNA object}{TReNA.Rdash.class}
\aliasA{.TReNA}{TReNA-class}{.TReNA}
%
\begin{Description}\relax
An S4 class to represent a TReNA object
\end{Description}
%
\begin{Arguments}
\begin{ldescription}
\item[\code{mtx.assay}] An assay matrix of gene expression data

\item[\code{solver}] A string matching the designated solver for relating a target gene to transcription factors. (default = "lasso")
\end{ldescription}
\end{Arguments}
%
\begin{Value}
An object of the TReNA class
\end{Value}
\inputencoding{utf8}
\HeaderA{VarianceFilter-class}{Filter based on gene expression variance}{VarianceFilter.Rdash.class}
\aliasA{.VarianceFilter}{VarianceFilter-class}{.VarianceFilter}
%
\begin{Description}\relax
Filter based on gene expression variance
\end{Description}
%
\begin{Arguments}
\begin{ldescription}
\item[\code{mtx.assay}] An assay matrix of gene expression data
\end{ldescription}
\end{Arguments}
\printindex{}
\end{document}
